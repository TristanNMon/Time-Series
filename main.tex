\documentclass[11pt]{article}
\usepackage{theme}
\usepackage{shortcuts}
% Document parameters
% Document title
\title{Assignment 1 (ML for TS) - MVA}
\author{
Firstname Lastname \email{youremail1@mail.com} \\ % student 1
Firstname Lastname \email{youremail2@mail.com} % student 2
}

\begin{document}
\maketitle

\section{Introduction}

\paragraph{Objective.} This assignment has three parts: questions about convolutional dictionary learning, spectral features, and a data study using the DTW. 

\paragraph{Warning and advice.} 
\begin{itemize}
    \item Use code from the tutorials as well as from other sources. Do not code yourself well-known procedures (e.g., cross-validation or k-means); use an existing implementation. 
    \item The associated notebook contains some hints and several helper functions.
    \item Be concise. Answers are not expected to be longer than a few sentences (omitting calculations).
\end{itemize}



\paragraph{Instructions.}
\begin{itemize}
    \item Fill in your names and emails at the top of the document.
    \item Hand in your report (one per pair of students) by Sunday 9\textsuperscript{th} November 23:59 PM.
    \item Rename your report and notebook as follows:\\ \texttt{FirstnameLastname1\_FirstnameLastname2.pdf} and\\ \texttt{FirstnameLastname1\_FirstnameLastname2.ipynb}.\\
    For instance, \texttt{LaurentOudre\_ValerioGuerrini.pdf}.
    \item Upload your report (PDF file) and notebook (IPYNB file) using this link: \footnotesize{LINK}.
\end{itemize}


\section{Convolution dictionary learning}

\begin{exercise}
Consider the following Lasso regression:
\begin{equation}\label{eq:lasso}
    \min_{\beta\in\RR^p} \frac{1}{2}\norm{y-X\beta}^2_2 \quad + \quad \lambda \norm{\beta}_1
\end{equation}
where $y\in\RR^n$ is the response vector, $X\in\RR^{n\times p}$ the design matrix, $\beta\in\RR^p$ the vector of regressors and $\lambda>0$ the smoothing parameter.

Show that there exists $\lambda_{\max}$ such that the minimizer of~\eqref{eq:lasso} is $\mathbf{0}_p$ (a $p$-dimensional vector of zeros) for any $\lambda > \lambda_{\max}$. 
\end{exercise}

\begin{solution}  % ANSWER HERE

\begin{equation}
    \lambda_{\max} = \dots
\end{equation}

\end{solution}

\begin{exercise}
For a univariate signal $\mathbf{x}\in\mathbb{R}^n$ with $n$ samples, the convolutional dictionary learning task amounts to solving the following optimization problem:

\begin{equation}
\min_{(\mathbf{d}_k)_k, (\mathbf{z}_k)_k \\ \norm{\mathbf{d}_k}_2^2\leq 1} \quad\norm{\mathbf{x} - \sum_{k=1}^K \mathbf{z}_k * \mathbf{d}_k }^2_2 \quad + \quad\lambda \sum_{k=1}^K \norm{\mathbf{z}_k}_1
\end{equation}

where $\mathbf{d}_k\in\mathbb{R}^L$ are the $K$ dictionary atoms (patterns), $\mathbf{z}_k\in\mathbb{R}^{N-L+1}$ are activations signals, and $\lambda>0$ is the smoothing parameter.

Show that
\begin{itemize}
    \item for a fixed dictionary, the sparse coding problem is a lasso regression (explicit the response vector and the design matrix);
    \item for a fixed dictionary, there exists $\lambda_{\max}$ (which depends on the dictionary) such that the sparse codes are only 0 for any $\lambda > \lambda_{\max}$. 
\end{itemize}
\end{exercise}

\begin{solution}  % ANSWER HERE
We fix all the dictionary atoms $\mathbf{d}_k \in \mathbb{R}^{L}$ for $k = 1, \dots, K$,
and we note that $M = N - L + 1$ is the valid convolution length. 

For a given pair $(\mathbf{z}_k, \mathbf{d}_k)$, the valid convolution 
$\mathbf{z}_k * \mathbf{d}_k \in \mathbb{R}^{N}$ can be written componentwise as
\[
(\mathbf{z}_k * \mathbf{d}_k)[n] = \sum_{i=0}^{M-1} z_k[i]\, d_k[n - i],
\quad \text{where } d_k[n - i] = 0 \text{ if } n - i < 0 \text{ or } n - i > L - 1.
\]

This operation can be expressed as a matrix–vector product
\[
\mathbf{y}_k = \mathbf{D}_k \mathbf{z}_k,
\]
where $\mathbf{D}_k \in \mathbb{R}^{N \times M}$ has coefficients
\[
\mathbf{D}_k[n, i] =
\begin{cases}
d_k[n - i], & \text{if } 0 \le n - i < L, \\[6pt]
0, & \text{otherwise.}
\end{cases}
\]

We define:
\[
\mathbf{z} =
\begin{bmatrix}
\mathbf{z}_1 \\ \vdots \\ \mathbf{z}_K
\end{bmatrix} \in \mathbb{R}^{K \cdot M},
\qquad
\mathbf{D} =
\begin{bmatrix}
\mathbf{D}_1 & \mathbf{D}_2 & \cdots & \mathbf{D}_K
\end{bmatrix} \in \mathbb{R}^{N \times K \cdot M}.
\]

Hence we can then rewrite the sparse coding problem given the $(\mathbf{d}_k)$ in the LASSO form, with 
response vector $\mathbf{x} \in \mathbb{R}^{N}$, design matrix $\mathbf{D}$ and regressor vector $\mathbf{z}$:
\[
\min_{\mathbf{z} \in \mathbb{R}^{K \cdot M}} \;
\left\| \mathbf{x} - \mathbf{D} \mathbf{z} \right\|_2^2
\;+\;
\lambda \|\mathbf{z}\|_1.
\]



\end{solution}

\section{Spectral feature}

Let $X_n$ ($n=0,\dots, N-1)$ be a weakly stationary random process with zero mean and autocovariance function $\gamma(\tau):= \mathbb{E}(X_n X_{n+\tau})$.
Assume the autocovariances are absolutely summable, \ie $\sum_{\tau\in\mathbb{Z}} |\gamma(\tau)| < \infty$, and square summable, \ie $\sum_{\tau\in\mathbb{Z}} \gamma^2(\tau) < \infty$.
Denote the sampling frequency by $f_s$, meaning that the index $n$ corresponds to the time $n / f_s$. For simplicity, let $N$ be even.


The \textit{power spectrum} $S$ of the stationary random process $X$ is defined as the Fourier transform of the autocovariance function:
\begin{equation}
    S(f) := \sum_{\tau=-\infty}^{+\infty}\gamma(\tau)e^{-2\pi f\tau/f_s}.
\end{equation}
The power spectrum describes the distribution of power in the frequency space. 
Intuitively, large values of $S(f)$ indicate that the signal contains a sine wave at the frequency $f$.
There are many estimation procedures to determine this important quantity, which can then be used in a machine-learning pipeline.
In the following, we discuss the large sample properties of simple estimation procedures and the relationship between the power spectrum and the autocorrelation.

(Hint: use the many results on quadratic forms of Gaussian random variables to limit the number of calculations.)

\begin{exercise}
In this question, let $X_n$ ($n=0,\dots,N-1)$ be a Gaussian white noise.

\begin{itemize}
    \item Calculate the associated autocovariance function and power spectrum. (By analogy with the light, this process is called ``white'' because of the particular form of its power spectrum.)
\end{itemize}

\end{exercise}

\begin{solution}

\end{solution}


\begin{exercise}
A natural estimator for the autocorrelation function is the sample autocovariance
\begin{equation}
    \hat{\gamma}(\tau) := (1/N) \sum_{n=0}^{N-\tau-1} X_n X_{n+\tau}
\end{equation}
for $\tau=0,1,\dots,N-1$ and $\hat{\gamma}(\tau):=\hat{\gamma}(-\tau)$ for $\tau=-(N-1),\dots,-1$.
\begin{itemize}
    \item Show that $\hat{\gamma}(\tau)$ is a biased estimator of $\gamma(\tau)$ but asymptotically unbiased.
    What would be a simple way to de-bias this estimator?
\end{itemize}

\end{exercise}

\begin{solution}

\end{solution}

\begin{exercise}
Define the discrete Fourier transform of the random process $\{X_n\}_n$ by
\begin{equation}
    J(f) := (1/\sqrt{N})\sum_{n=0}^{N-1} X_n e^{-2\pi\iu f n/f_s}
\end{equation}
The \textit{periodogram} is the collection of values $|J(f_0)|^2$, $|J(f_1)|^2$, \dots, $|J(f_{N/2})|^2$ where $f_k = f_s k/N$.
(They can be efficiently computed using the Fast Fourier Transform.)
\begin{itemize}
    \item Write $|J(f_k)|^2$ as a function of the sample autocovariances.
    \item For a frequency $f$, define $f^{(N)}$ the closest Fourier frequency $f_k$ to $f$.
    Show that $|J(f^{(N)})|^2$ is an asymptotically unbiased estimator of $S(f)$ for $f>0$.
\end{itemize}
\end{exercise}

\begin{solution}
\[
|J(f_k)|^2 = J(f_k) \overline{J(f_k)} 
= (1/N) \sum_{n=0}^{N-1} \sum_{m=0}^{N-1} X_n X_m e^{-2\pi\iu f_k (n - m)/f_s}
\]

Do the change of variable $\tau = m - n \in [-(N-1), N-1]$.
\[
|J(f_k)|^2 = \frac{1}{N} \sum_{\tau=-(N-1)}^{N-1} 
    e^{-2\pi \mathrm{i} f_k \tau / f_s}
    \sum_{n=0}^{N-1 - |\tau|} X_n X_{n + |\tau|} \\[4pt]
\]

\[
|J(f_k)|^2= \sum_{\tau=-(N-1)}^{N-1} 
    \hat{\gamma}(\tau)\, e^{-2\pi \mathrm{i} f_k \tau / f_s}.
\]

For a frequency $f$, let $f^{(N)}$ be the closest Fourier frequency to $f$,
with $k_n = \lfloor N f / f_s \rfloor$.

\[
\mathbb{E}[|J(f^{(N)})|^2]
= \sum_{\tau=-(N-1)}^{N-1} 
    \mathbb{E}[\hat{\gamma}(\tau)]\, e^{-2\pi \mathrm{i} f^{(N)} \tau / f_s}
\]

\[
\mathbb{E}[|J(f^{(N)})|^2]
= \frac{N-|\tau|}{N} \sum_{\tau=-(N-1)}^{N-1} 
     \gamma(\tau)\, e^{-2\pi \mathrm{i} f^{(N)} \tau / f_s}
\]

TODO We assume the summability of the autocovariances and their square is sufficient
to take the limit of $N$ in the whole expression, to have

\[
\mathbb{E}[|J(f^{(N)})|^2] \rightarrow \sum_{\tau=-\infty}^{+\infty} 
     \gamma(\tau)\, e^{-2\pi \mathrm{i} f \tau / f_s}
= S(f).
\]


\end{solution}

\begin{exercise}\label{ex:wn-exp}
    In this question, let $X_n$ ($n=0,\dots,N-1)$ be a Gaussian white noise with variance $\sigma^2=1$ and set the sampling frequency to $f_s=1$ Hz
    \begin{itemize}
        \item For $N\in\{200, 500, 1000\}$, compute the \textit{sample autocovariances} ($\hat{\gamma}(\tau)$ vs $\tau$) for 100 simulations of $X$.
        Plot the average value as well as the average $\pm$, the standard deviation.
        What do you observe?
        \item For $N\in\{200, 500, 1000\}$, compute the \textit{periodogram} ($|J(f_k)|^2$ vs $f_k$) for 100 simulations of $X$.
        Plot the average value as well as the average $\pm$, the standard deviation.
        What do you observe?
    \end{itemize}
    Add your plots to Figure~\ref{fig:wn-exp}.
    
\begin{figure}
    \centering
    \begin{minipage}[t]{0.3\textwidth}
    \centerline{\includegraphics[width=\textwidth]{figures/acov_white_noise_N200_fs1.0_sim100.png}}
    \centerline{Autocovariance ($N=200$)}
    \end{minipage}
    \begin{minipage}[t]{0.3\textwidth}
    \centerline{\includegraphics[width=\textwidth]{figures/acov_white_noise_N500_fs1.0_sim100.png}}
    \centerline{Autocovariance ($N=500$)}
    \end{minipage}
    \begin{minipage}[t]{0.3\textwidth}
    \centerline{\includegraphics[width=\textwidth]{figures/acov_white_noise_N1000_fs1.0_sim100.png}}
    \centerline{Autocovariance ($N=1000$)}
    \end{minipage}
    \vskip1em
    \begin{minipage}[t]{0.3\textwidth}
    \centerline{\includegraphics[width=\textwidth]{figures/periodogram_white_noise_N200_fs1.0_sim100.png}}
    \centerline{Periodogram ($N=200$)}
    \end{minipage}
    \begin{minipage}[t]{0.3\textwidth}
    \centerline{\includegraphics[width=\textwidth]{figures/periodogram_white_noise_N500_fs1.0_sim100.png}}
    \centerline{Periodogram ($N=500$)}
    \end{minipage}
    \begin{minipage}[t]{0.3\textwidth}
    \centerline{\includegraphics[width=\textwidth]{figures/periodogram_white_noise_N1000_fs1.0_sim100.png}}
    \centerline{Periodogram ($N=1000$)}
    \end{minipage}
    \vskip1em
    \caption{Autocovariances and periodograms of a Gaussian white noise (see Question~\ref{ex:wn-exp}).}
    \label{fig:wn-exp}
\end{figure}

\end{exercise}

\begin{solution}
    \begin{itemize}
        \item The theoretical autocovariance of white noise is $\sigma^2$ at lag=0 and 0 everywhere else. 
        Here the sample autocovariance has the peak of 1 at lag=0 and abruptly decreases to low fluctuations around 0 with standard deviation of ~0.1 for N=200.
        The standard deviation slowly decreases with the number of lags. That is coherent with the expression of the variance derived below. With bigger N, the standard deviation is lower for all lags, around 0.02 for N=500 at low lag.
        \item Theoretically, the periodogram of gaussian white noise is flat and equal to $\sigma^2/f_s = 1$ at all frequencies.
        Here we have a mean around 1 and a standard deviation of 1 too (we rescaled the autocovariances by 0.5 after computing the onesided periodogram)
        The standard deviation does not decrease with N.
    \end{itemize}
Expression of variance:
\[
\mathrm{var}\big(\hat{\gamma}(\tau)\big)
= \frac{1}{N^2}\sum_{n=0}^{N-1-\tau} \mathrm{var}(X_n X_{n+\tau})
= \frac{N-\tau}{N^2}\,\mathbb{E}[X_0^2 X_{0+\tau}^2] = \frac{N-\tau}{N^2}\,\sigma^4.
\]

Therefore,
\[
\mathrm{std}\big(\hat{\gamma}(\tau)\big)
= \sigma^2 \sqrt{\frac{1-\tau/N}{N}}.
\]
which decreases to \(0\) as \(\tau \rightarrow N\).

 


\end{solution}

\begin{exercise}
    We want to show that the estimator $\hat{\gamma}(\tau)$ is consistent, \ie it converges in probability when the number $N$ of samples grows to $\infty$ to the true value ${\gamma}(\tau)$.
    In this question, assume that $X$ is a wide-sense stationary \textit{Gaussian} process.
    \begin{itemize}
        \item Show that for $\tau>0$ 
    \begin{equation}
       \text{var}(\hat{\gamma}(\tau)) = (1/N) \sum_{n=-(N-\tau-1)}^{n=N-\tau-1} \left(1 - \frac{\tau + |n|}{N}\right) \left[\gamma^2(n) + \gamma(n-\tau)\gamma(n+\tau)\right].
    \end{equation}
    (Hint: if $\{Y_1, Y_2, Y_3, Y_4\}$ are four centered jointly Gaussian variables, then $\mathbb{E}[Y_1 Y_2 Y_3 Y_4] = \mathbb{E}[Y_1 Y_2]\mathbb{E}[Y_3 Y_4] + \mathbb{E}[Y_1 Y_3]\mathbb{E}[Y_2 Y_4] + \mathbb{E}[Y_1 Y_4]\mathbb{E}[Y_2 Y_3]$.) 
    \item Conclude that $\hat{\gamma}(\tau)$ is consistent.
    \end{itemize}
\end{exercise}

\begin{solution}
    
\end{solution}

Contrary to the correlogram, the periodogram is not consistent.
It is one of the most well-known estimators that is asymptotically unbiased but not consistent.
In the following question, this is proven for Gaussian white noise, but this holds for more general stationary processes.
\begin{exercise}
    Assume that $X$ is a Gaussian white noise (variance $\sigma^2$) and let $A(f):=\sum_{n=0}^{N-1} X_n \cos(-2\pi f n/f_s)$ and $B(f):=\sum_{n=0}^{N-1} X_n \sin(-2\pi f n/f_s$.
    Observe that $J(f) = (1/N) (A(f) + \iu B(f))$.
    \begin{itemize}
        \item Derive the mean and variance of $A(f)$ and $B(f)$ for $f=f_0, f_1,\dots, f_{N/2}$ where $f_k=f_s k/N$.
        \item What is the distribution of the periodogram values $|J(f_0)|^2$, $|J(f_1)|^2$, \dots, $|J(f_{N/2})|^2$.
        \item What is the variance of the $|J(f_k)|^2$? Conclude that the periodogram is not consistent.
        \item Explain the erratic behavior of the periodogram in Question~\ref{ex:wn-exp} by looking at the covariance between the $|J(f_k)|^2$.
    \end{itemize}
\end{exercise}

\begin{solution}

1) $A(f)$ and $B(f)$ are linear combinations of centered independent Gaussian variables, hence
they are also centered Gaussian:
\[
\mathbb{E}[A(f)] = 0, \quad \mathbb{E}[B(f)] = 0.
\]

For the variance, by bilinearity of covariances:
\begin{align*}
\mathrm{var}\big(A(f)\big) 
&= \mathrm{cov}\!\left( 
    \sum_{n=0}^{N-1} X_n \cos(2\pi f n/f_s),
    \sum_{m=0}^{N-1} X_m \cos(2\pi f m/f_s)
\right) \\
&= \sum_{n=0}^{N-1} \sum_{m=0}^{N-1} 
    \mathrm{cov}(X_n, X_m)\,
    \cos(2\pi f n/f_s)
    \cos(2\pi f m/f_s) \\
&= \sigma^2 \sum_{n=0}^{N-1} 
    \cos^2(2\pi f n/f_s).
\end{align*}

For $k \in \{1, \ldots, N/2-1\}$, we have:
\begin{align*}
\mathrm{var}\big(A(f_k)\big) 
&= \sigma^2 \sum_{n=0}^{N-1} 
    \cos^2\!\left(\frac{2\pi k n}{N/f_s}\right) \\
&= \sigma^2 \sum_{n=0}^{N-1} 
    \frac{1}{2}\left(1 - \cos\!\left(\frac{4\pi k n}{N/f_s}\right)\right) \\
&= \frac{N\sigma^2}{2}.
\end{align*}

The same result holds for $B(f_k)$.

For $k = 0$ or $k = N/2$, we have $\cos^2(2\pi k n/(N/f_s)) = 1$ and $\sin^2(2\pi k n/(N/f_s)) = 0$, so:
\[
\mathrm{var}\big(A(f_0)\big) = \mathrm{var}\big(A(f_{N/2})\big) = N\sigma^2, 
\quad \mathrm{var}\big(B(f_0)\big) = \mathrm{var}\big(B(f_{N/2})\big) = 0.
\]

To sum up:
\[
\operatorname{Var}[A(f_k)] =
\begin{cases}
    N\sigma^2, & \text{if } k = 0 \text{ or } k = N/2, \\[6pt]
    (N/2)\sigma^2, & \text{if } 0 < k < N/2,
\end{cases}
\qquad
\operatorname{Var}[B(f_k)] =
\begin{cases}
    0, & \text{if } k = 0 \text{ or } k = N/2, \\[6pt]
    (N/2)\sigma^2, & \text{if } 0 < k < N/2.
\end{cases}
\]

---

2) The periodogram writes:
\[
|J(f_k)|^2 = \frac{1}{N} \big(A(f_k)^2 + B(f_k)^2\big).
\]

For $0 < k < N/2$:
\begin{align*}
|J(f_k)|^2
    &= \frac{1}{N} \left((N/2)\sigma^2 Z_A^2 + (N/2)\sigma^2 Z_B^2\right) \\
    &= \frac{\sigma^2}{2} (Z_A^2 + Z_B^2),
\end{align*}
where $Z_A, Z_B \sim \mathcal{N}(0, 1)$ are independent, since their covariance is zero (using trigonometric orthogonality):
\begin{align*}
\mathrm{cov}(A(f_k), B(f_k)) 
&= \sum_{n=0}^{N-1} \sum_{m=0}^{N-1} 
    \mathrm{cov}(X_n, X_m)\,
    \cos(2\pi f_k n/f_s)
    \sin(2\pi f_k m/f_s) \\
&= \sigma^2 \sum_{n=0}^{N-1}
    \frac{1}{2}\sin(4\pi f_k n/f_s) = 0.
\end{align*}

Hence
\[
|J(f_k)|^2 = \frac{\sigma^2}{2}(Z_A^2 + Z_B^2)
    \sim \frac{\sigma^2}{2}\chi^2_2.
\]

For $k = 0$ or $k = N/2$, we have $B(f_k) = 0$ and 
$A(f_k) \sim \mathcal{N}(0, N\sigma^2)$, hence
\[
|J(f_k)|^2 = \frac{1}{N} A(f_k)^2 \sim \sigma^2 \chi^2_1.
\]

---

3) For $0 < k < N/2$:
\begin{align*}
\operatorname{Var}[|J(f_k)|^2]
&= \operatorname{Var}\!\left[\frac{\sigma^2}{2}\chi^2_2\right] \\
&= \frac{\sigma^4}{4} \operatorname{Var}[\chi^2_2] \\
&= \frac{\sigma^4}{4} \cdot (2 \cdot 2) = \sigma^4.
\end{align*}

And for $k = 0$ or $k = N/2$:
\begin{align*}
\operatorname{Var}[|J(f_k)|^2]
&= \operatorname{Var}[\sigma^2 \chi^2_1] \\
&= \sigma^4 \operatorname{Var}[\chi^2_1] \\
&= \sigma^4 \cdot 2 = 2\sigma^4.
\end{align*}
Because the variance is constant and does not converge to 0 with $N$ we 
believe the periodogram is not consistent, ie does not converge in probability to the true power spectrum.
We don't provide a rigorous proof here.

4)
The covariance between two periodogram values at different frequencies $f_k$ and $f_{l}$, with $k \neq l$, is $0$.
We don't prove it but it is similar calculation as for $A(f_k)$ and $B(f_k)$ not correlated and independent as gaussian,
but here we take $A^2(f_k)$, $B^2(f_k)$, $A^2(f_l)$ and $B^2(f_l)$.
Hence the periodogram value at one frequency is completely not correlated with the value at the next frequency.
It explains the erratic behavior and noisy and spiky look, rather than a smooth curve.




\end{solution}


\begin{exercise}\label{q:barlett}
    As seen in the previous question, the problem with the periodogram is the fact that its variance does not decrease with the sample size.
    A simple procedure to obtain a consistent estimate is to divide the signal into $K$ sections of equal durations, compute a periodogram on each section, and average them.
    Provided the sections are independent, this has the effect of dividing the variance by $K$. 
    This procedure is known as Bartlett's procedure.
    \begin{itemize}
        \item Rerun the experiment of Question~\ref{ex:wn-exp}, but replace the periodogram by Barlett's estimate (set $K=5$). What do you observe?
    \end{itemize}
    Add your plots to Figure~\ref{fig:barlett}.
\end{exercise}

\begin{solution}
The Bartlett method to compute periodogram reduces the variance by $1/K$:
\[\operatorname{Var}[\hat{S}_\text{Bartlett}(f_k)] = \frac{1}{K} \operatorname{Var}[\hat{S}(f_k)]\]
As the periodograms on the non overlapping sections are independent. 
Hence it reduces the standard deviation by $1/\sqrt{K} \sim 0.45$, as we see in Figure~\ref{fig:barlett},

\begin{figure}
    \centering
    \begin{minipage}[t]{0.3\textwidth}
    \centerline{\includegraphics[width=\textwidth]{figures/bartlett_periodogram_white_noise_N200_K5_fs1.0_sim100.png}}
    \centerline{Periodogram ($N=200$)}
    \end{minipage}
    \begin{minipage}[t]{0.3\textwidth}
    \centerline{\includegraphics[width=\textwidth]{figures/bartlett_periodogram_white_noise_N500_K5_fs1.0_sim100.png}}
    \centerline{Periodogram ($N=500$)}
    \end{minipage}
    \begin{minipage}[t]{0.3\textwidth}
    \centerline{\includegraphics[width=\textwidth]{figures/bartlett_periodogram_white_noise_N1000_K5_fs1.0_sim100.png}}
    \centerline{Periodogram ($N=1000$)}
    \end{minipage}
    \vskip1em
    \caption{Barlett's periodograms of a Gaussian white noise (see Question~\ref{q:barlett}).}
    \label{fig:barlett}
\end{figure}

\end{solution}
\section{Data study}

\subsection{General information}

\paragraph{Context.}
The study of human gait is a central problem in medical research with far-reaching consequences in the public health domain. This complex mechanism can be altered by a wide range of pathologies (such as Parkinson's disease, arthritis, stroke,\ldots), often resulting in a significant loss of autonomy and an increased risk of falls. Understanding the influence of such medical disorders on a subject's gait would greatly facilitate early detection and prevention of those possibly harmful situations. To address these issues, clinical and bio-mechanical researchers have worked to objectively quantify gait characteristics.

Among the gait features that have proved their relevance in a medical context, several are linked to the notion of step (step duration, variation in step length, etc.), which can be seen as the core atom of the locomotion process. Many algorithms have, therefore, been developed to automatically (or semi-automatically) detect gait events (such as heel-strikes, heel-off, etc.) from accelerometer and gyrometer signals.

\paragraph{Data.}
Data are described in the associated notebook.

\subsection{Step classification with the dynamic time warping (DTW) distance}

\paragraph{Task.} The objective is to classify footsteps and then walk signals between healthy and non-healthy.

\paragraph{Performance metric.} The performance of this binary classification task is measured by the F-score.


\begin{exercise}
Combine the DTW and a k-neighbors classifier to classify each step. Find the optimal number of neighbors with 5-fold cross-validation and report the optimal number of neighbors and the associated F-score. Comment briefly.
\end{exercise}

\begin{solution}

\end{solution}

\newpage
\begin{exercise}\label{q:class-errors}
Display on Figure~\ref{fig:class-errors} a badly classified step from each class (healthy/non-healthy).
\end{exercise}

\begin{solution}
\begin{figure}
    \centering
    \begin{minipage}[t]{\textwidth}
    \centerline{\includegraphics[width=0.6\textwidth]{example-image-golden}}
    \centerline{Badly classified healthy step}
    \end{minipage}
    \vskip1em
    \begin{minipage}[t]{\textwidth}
    \centerline{\includegraphics[width=0.6\textwidth]{example-image-golden}}
    \centerline{Badly classified non-healthy step}
    \end{minipage}
    \vskip1em
    \caption{Examples of badly classified steps (see Question~\ref{q:class-errors}).}
    \label{fig:class-errors}
\end{figure}
\end{solution}


\end{document}
